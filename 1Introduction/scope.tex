\subsection{Project Scope and Applications}
Our project aims to address the prevalent issue of false videos in the realm of deepfake technology. Due to the limited availability of reliable detection tools, our main focus is on creating an accessible and user-friendly deepfake detection software. The primary objective is to curb the widespread dissemination of deepfakes by offering users a platform where they can easily upload digital content and distinguish between authentic and manipulated materials. Our goal is to contribute to a vigilant digital environment, promoting trust and authenticity in the face of evolving challenges posed by deceptive media content.

\noindent Our project helps in detecting and addressing deepfake content. However, it's crucial to recognize certain limitations. The effectiveness of our system  can be influenced by multiple factors like the characteristics of the input data. The detection of sequence of frames in videos cannot be tracked and the audio detection remains out of our scope. Also, the inputs with tikas and face-masks are difficult to classify.\\
\newpage
\noindent \textbf{Potential Application}

\begin{itemize}
    \item \textbf{Cybersecurity:} It can be integrated into cybersecurity systems to identify and prevent the spread of manipulated media that may pose security threats or deceive users.

    \item \textbf{Governmental Organizations:} Adoption within governmental organizations to safeguard against deepfake threats, especially in areas concerning national security, public figures, and official communications.

    \item \textbf{News and Journalism:} Integration into newsrooms and journalistic processes to verify media content, ensuring the dissemination of accurate information and maintaining the integrity of news reporting.

    \item \textbf{Education and Research:} Utilization in educational institutions for media literacy programs and research purposes, empowering students and researchers to critically assess the authenticity of visual content.
\end{itemize}