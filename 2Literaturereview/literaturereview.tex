\section{LITERATURE REVIEW}

The population of pets kept in households have increased worldwide recently. It is reported that
43\% of families own dogs or cats in Italy [1].

\noindent There have been several researches to make computer-aided decision-making systems for cattle~\textquoteright s. health monitoring as the form of expert system [3, 13-17]. Those systems can handle a limited range
of diseases for a certain cattle such as dog [3], horse [13], cow [14], fish [15], swine [16], and general cattle
[17]. Usually their data collection method is primary and only can handle less than 10 related diseases with
traditional certainty factor [13-16] or fuzzy logic [17]. Thus the utility of such system is very limited
and possibly unreliable or subjective and bears a model accuracy of 72\%.

\noindent In our case however,  we utilized the MPL classifier to predict diseases in canines based on visible symptoms. Our analysis demonstrated that this algorithm surpassed all other methods examined in the research. And bears a model accuracy of 82\% which is greater then the mentioned model and our model handle more than 50 related diseases with 166 symptoms and bears accuracy of 82\%. 


\noindent There are several other systems that use content-based filtering to help
users find information on the World Wide Web. Letizia [23] is a user
interface that assists users browsing the web. The system tracks the browsing
behavior of a user and tries to anticipate what pages may be of interest to the user.
Syskill \& Webert [24] is a software agent that tries to determine
which web pages might interest a user by using a naive Bayesian classifier. A user
provides training instances by rating explored pages as either hot or cold. Jennings
and Higuchi [25] describe a neural network that models the
interests of a user in a Usenet news environment. The neural network is formed and
modified as a result of the articles a user has read or rejected.

\noindent Our system uses a straightforward item recommendation approach to users, steering clear of complexity and ensuring speed and reliability through the utilization of a simple cosine similarity algorithm for content-based filtering.


\noindent Dog Breed Identification Using Deep Learning study used deep learning and
CNNs for breed identification, similar to HelloPet’s approach. The architecture is based
on image classification using TensorFlow. This method reported an accuracy of 91.45\% for breed classification using Deep Learning. HelloPet’s breed detection model, which
likely uses a similar deep learning approach, outperforms this method with a higher
accuracy of 95.47\%. The strength lies in leveraging advanced technologies like deep
learning, but the lack of detailed results and a broader comparison with other methods is
a limitation [8].
Dog Breed Classifier using Convolutional Neural Networks,
the authors focus on a deep learning approach for dog breed identification, utilizing
transfer learning with pre-trained CNNs. This method reported an accuracy of 89.52\%
for breed classification using CNNs. HelloPet’s breed detection model, which likely uses
a similar deep learning approach, outperforms this method with a higher accuracy of
95.47\%. While the quantitative result is impressive, the architecture’s detailed methodology is not explicitly outlined, limiting the comprehensive understanding of the model.The strength lies in achieving high accuracy, but the lack of architectural details and limited result discussion are notable weaknesses [9]
